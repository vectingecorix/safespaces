\documentclass[]{article}
\usepackage{amsmath}
\usepackage{amsfonts}
\usepackage{amssymb}
\usepackage{amsthm}
\usepackage[utf8]{inputenc}
\usepackage[T1]{fontenc}
\usepackage{thmtools}
\usepackage{mathtools}
\usepackage{enumerate}


\newtheorem{theorem}{Theorem}
\newtheorem{definition}{Definition}
\newtheorem{corollary}{Corollary}
\newtheorem{lemma}{Lemma}


\begin{document}

\title{Safe Spaces: An Introductory Course}

\maketitle 

\begin{abstract}
Safe spaces are key icons of modern political debates. However, from a mathematical point of view the subject lacks rigorous studies. In this paper, we aim for a better understanding of safe spaces by providing exact definitions of safety inspired by linear algebra and analysis. 
\end{abstract}

\section{Introduction}



\section{Preliminaries}

\begin{definition}[emancipated, safe spaces, unsafe spaces]
	Let $K$ be a field and $V$ be a $K$-vector space. We call $v \in V$ \emph{emancipated} if for every endomorphism $\alpha: V \to V$ and every $w \in V$ it holds that $\alpha(v) = \alpha(w)$. Further, we call $V$ a \emph{$K$-safe space} (or \emph{safe space} for short) if $V$ contains only emancipated vectors. Otherwise, $V$ is an \emph{unsafe space}.
\end{definition}


\begin{definition}[integrable]
	Let $K$ be a field, $V$ be a $K$-safe space, and $W$ be a unsafe space.
	A vector $w \in W$ is called \emph{integrable} if $V \cup \{w\}$ is a $K$-safe space.
\end{definition}

\begin{definition}[relatively safe spaces]
    Let $K$ be a field, $V$ be a $K$-vector space and $U, W \subseteq V $ not
    necessarily safe subspaces. $U$ and $W$ are called \emph{relatively safe}, if
    and only if $U \cap W$ is a safe space.
\end{definition}

\begin{definition}[filter bubble, minimal filter bubble, foam bath]
    Let $K$ be a field, $V$ be a $K$-vector space, $U \subseteq V$ a subspace
    and $\alpha: V \to V$ $K$-linear. We call $U$ an
    \emph{$\alpha$-filter bubble}, if $\alpha(U) \subseteq U$. In Addition, if
    there is no unsafe filter bubble $W \subsetneq U$, $U$ is called a \emph{minimal
        filter bubble}. A pairwise relatively safe family $(U_i)_{i \in \mathbb
    N}$ of minimal $\alpha$-filter bubbles is called a \emph{foam bath}, if
    \[
        \bigcup_{i = 1}^\infty U_i = V
    \]
\end{definition}


\section{Results}

\begin{theorem}\label{thm:safe-space-trivial}
	Let $V$ be a $K$-safe space. Then $V$ is trivial.
\end{theorem}
\begin{proof}
	Assume that $V$ is not trivial, i.e. there exists some $v \neq 0$ in $V$. Then for $\alpha := \text{id}$ we get $\alpha(v) = v \neq 0 = \alpha(0)$. Hence, $V$ is an unsafe space.
\end{proof}

\begin{corollary}\label{cor:safe-space-dim0}
	Let $K$ be a field and $V$ be a vector space. Then $V$ is a safe space if and only if $\dim V = 0$.
\end{corollary}
The proof is left as an exercise to interested readers as it follows directly from Theorem~\ref{thm:safe-space-trivial}.

\begin{corollary}
	Let $K$ be a field and $V, W$ $K$-safe spaces. Then $V$ and $W$ are isomorphic.
\end{corollary}

\begin{theorem}
	Let $K$ be a field, $V$ be some finite-dimensional $K$-vector space and $\varphi : V \to V$ a $K$-linear function. Then the following statements are equivalent:
	\begin{enumerate}[(i)]
		\item $\varphi$ is bijective.
		\item $\varphi$ is injective.
		\item $\varphi$ is surjective.
		\item $\ker(\varphi)$ is a safe space.
	\end{enumerate}
\end{theorem}

The proof is left as an exercise to the reader.

\begin{theorem}
	Let $V$ be a safe space. Then $V$ is closed under integration.

\end{theorem}

\begin{proof}
	Let $W$ be an unsafe space. For every $w \in W$ the following implication holds: If $w$ is integrable then $w = 0 \in V$. Therefore, $V$ is closed under integration.
\end{proof}

\begin{lemma}
	Orthogonal spaces are relatively safe, but not every pair of relatively safe spaces is orthogonal.
\end{lemma}

\begin{proof}
	Consider two orthogonal spaces $U$ and $W$, i.e. it holds that $\langle u, w \rangle = 0$ for all $u \in U$ and all $w \in W$. As $U$ and $W$ are orthogonal, $U \cap W = \{0\}$. Otherwise there existed some $v \in U \cap W$, $v \neq 0$, such that $\langle v, v \rangle > 0$.
	
	The other direction is not necessarily true. For example two linear independent real vectors $u \neq w$ with $\langle u,w \rangle \neq 0$ yield the subspaces $U = \{ \lambda u : \lambda \in \mathbb{R} \}$ and $W = \{ \lambda w : \lambda \in \mathbb{R}\}$. Obviously $U \cap W = \{0\}$, but $U$ and $W$ are not orthogonal by definition.
\end{proof}



\section{Conclusions}




\end{document}
